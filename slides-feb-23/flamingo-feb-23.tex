\documentclass{beamer}

\usepackage{listings}
\lstset{basicstyle=\ttfamily}

\begin{document}
\title{Flamingo Overview}   
\author{Jonathan Chan, Alex Ozer} 
\date{Thursday, February 23, 2017} 

\frame{\titlepage} 


\begin{frame}
\frametitle{Introduction}
\begin{itemize}
\item Mission code is almost entirely:
\begin{itemize}
\item bookkeeping,
\item logging, or
\item logic.
\end{itemize} 
\item Much more bookeeping and logging than logic. 
\item Errors in bookkeeping (and logging) are bad, and 
\item lots of the first two make critical logic hard to see.
\end{itemize}
\end{frame}


\begin{frame}[fragile]
\frametitle{Example: Logic}
\tiny
\lstset{moredelim=**[is][\color{red}]{@}{@}}
\begin{lstlisting}
covered_cutout = self.get_covered_cutout()
@if covered_cutout == None or self.num_tries['remove_cover'] >= 2:@
    self.state = self.State.target_uncovered_cutout

    if Torpedoes.lost_target:
        Torpedoes.lost_target = False
    else:
        Torpedoes.target = self.get_uncovered_cutout()

    @if self.target == None:@
        self.logi("Can't find a target cutout! Falling back to positional targeting")
        self.state = self.State.target_cutout_fallback
        @self.task = TargetCutoutFallback()@
    else:
        @self.task = ShootCutout()@
else:
    Torpedoes.target = covered_cutout
    self.state = self.State.remove_cover
    @self.task = RemoveCover()@

\end{lstlisting}
\end{frame}


\begin{frame}[fragile]
\frametitle{Example: Logic + Bookkeeping}
\tiny
\lstset{moredelim=**[is][\color{red}]{@}{@}}
\lstset{moredelim=**[is][\color{blue}]{?}{?}}
\begin{lstlisting}
?covered_cutout = self.get_covered_cutout()?
@if covered_cutout == None or self.num_tries['remove_cover'] >= 2:@
    ?self.state = self.State.target_uncovered_cutout

    if Torpedoes.lost_target:
        Torpedoes.lost_target = False
    else:
        Torpedoes.target = self.get_uncovered_cutout()?

    @if self.target == None:@
        self.logi("Can't find a target cutout! Falling back to positional targeting")
        ?self.state = self.State.target_cutout_fallback?
        @self.task = TargetCutoutFallback()@
    @else:@
        @self.task = ShootCutout()@
@else:@
    ?Torpedoes.target = covered_cutout
    self.state = self.State.remove_cover?
    @self.task = RemoveCover()@

\end{lstlisting}
\end{frame}


\begin{frame}
\frametitle{Example: Explanation}
As written:
\begin{itemize}
\item if there are no covered cutouts or we've tried to remove the cover more than twice,
\item if we can't find a target cutout,
\begin{itemize}
\item fall back to positional targeting to target a cutout (i.e. just shoot two torpedoes blindly),
\item otherwise try to target the cutout we've found and then shoot at it,
\end{itemize}
\item otherwise, try to remove a cover.
\end{itemize}


Alternatively:
``Find an uncovered cutout (following some order) and shoot at it.
If there is an covered cutout you can uncover it by doing $X$.
Alternatively, you can shoot blindly (which is less likely to succeed).''

Note that the order is inverted in general.
\end{frame}


\begin{frame}[fragile]
\frametitle{Extrapolating from the example}
\begin{itemize}
\item Most logic is pretty mechanical:
\begin{lstlisting}
if (covered_cutout == None or
    self.num_tries['remove_cover'] >= 2):
\end{lstlisting}
\item Most bookkeeping (\verb|self.x = e|) is either:
\begin{itemize}
\item passing things on to the next state (i.e. making up for not being able to do \verb|state.first().second()|), 
\item setting the state, or
\item setting up retry logic.
\end{itemize}
\item Logging ``can't find a target cutout'' inside of:
\begin{lstlisting}
if self.target == None:
    ...
\end{lstlisting}
... has to be redundant in some sense.
\item Idea: have the computer do as much of this as is possible for us!
\end{itemize}
\end{frame}

\begin{frame}[fragile]
\frametitle{Flamingo example}
\lstset{moredelim=**[is][\color{black}]{@}{@}}
\lstset{moredelim=**[is][\color{black}]{?}{?}}
\lstset{moredelim=**[is][\color{black}]{\$}{\$}}
\lstset{moredelim=**[is][\color{black}]{!}{!}}
\lstset{moredelim=**[is][\color{black}]{#}{#}}
\lstset{escapeinside={(*}{*)}}
\begin{lstlisting}
(*\textit{goal}*) "Shot small cutouts in order."
  (*\textit{do}*)
    as(?shot?[(*\underline{SmallCutoutW}*)],
       @before@(?shot?[(*\underline{SmallCutoutN}*)] !and!
              @forall@((*\underline{C}*) cutout,
                     @not@(?shot?[(*\underline{C}*)]) !or!
                     (*\underline{C}*) = (*\underline{SmallCutoutN}*))))
  (*\textit{for}*) 3000
\end{lstlisting}
\end{frame}


\begin{frame}[fragile]
\frametitle{Flamingo example (with more symbols)}
\lstset{basicstyle=\sffamily,columns=flexible,escapeinside={(*}{*)}}}
\begin{lstlisting}
(*\textit{goal}*) "Shot small cutouts in order."
  (*\textit{do}*)
    (*$\sim$*)((*\texttt{shot}*)[(*\textbf{SmallCutoutW}*)],
       (*$\uparrow$*)((*\texttt{shot}*)[(*\textbf{SmallCutoutN}*)] &
         (*$\forall$*) (*\textbf{C}*) cutout. (*$\lnot$*)(*\texttt{shot}*)[(*\textbf{C}*)] (*$\vee$*) (*\textbf{C}*) = (*\textbf{SmallCutoutN}*)))
  (*\textit{for}*) 3000
\end{lstlisting}
\end{frame}

% Ha ha...
\begin{frame}
\frametitle{What is Flamingo?}
\includegraphics[width=\textwidth]{arguably-alive.png}
\end{frame}

\begin{frame}
\frametitle{Shakey the Robot}
\begin{center}
\includegraphics[height=3in]{shakey.jpg}
\end{center}
\end{frame}

\begin{frame}
\frametitle{What is Flamingo?}
\begin{itemize}
\item A STRIPS-like automated planner with some nifty features.
\item Multiple components working together:
\begin{itemize}
\item \textbf{Planner}
\begin{itemize}
\item \textbf{Mission planner}: given some high level goals with different point values, requirements, probabilities of success, and a time limit, what's the plan that gets us as many points as possible?  (some subtlety here: more on this later)

\item \textbf{Motion planner}: given some position and orientation relative to some object, what's the best way to achieve that?

\end{itemize}
\item \textbf{Mission controller}: calls planners and adapts to action failures.

\item \textbf{Knowledge base}: what do we think we know about the world?
\end{itemize}

\item Goal is to create something that will last 4+ years, so careful design is important.
\end{itemize}
\end{frame}

\begin{frame}
\frametitle{Plan for this presentation}
\begin{itemize}
\item Not enough time to talk about everything.

\item A high-level overview of core ideas and potential difficulties.
\end{itemize}
\end{frame}

\begin{frame}[fragile]
\frametitle{Simple Predicates}
\lstset{moredelim=**[is][\color{red}]{@}{@}}
\lstset{escapeinside={(*}{*)}}
\begin{lstlisting}
(*\textit{goal}*) "Shot small cutouts in order."
  (*\textit{do}*)
    before(@shot[(*\underline{SmallCutoutN}*)@] and
           forall((*\underline{C}*) cutout, not(@shot[(*\underline{C}*)]@) or
                  (*\underline{C}*) = (*\underline{SmallCutoutN}*))) and
    @shot[(*\underline{SmallCutoutW}*)]@
  (*\textit{for}*) 3000
\end{lstlisting}
\end{frame}

\begin{frame}[fragile]
\frametitle{Simple Predicates}
\begin{itemize}
\item Examples:
\begin{itemize}
\item \verb|shot[SmallCutoutN]|: ``We shot at \texttt{SmallCutoutN} at some point in the past.''
\item \verb|grabbing[VertGreenStack]|: ``We are grabbing \texttt{VertGreenStack}.''
\end{itemize}
\item A \textit{predicate} is a symbol (e.g. \verb|shot|, \verb|grabbed|), and some tuple of argument types (the predicate's type).
\item An \textit{atomic formula} is made up a predicate symbol followed by some number of expressions corresponding to the type of the predicate.
\item At any given time, what the mission controller ``knows'' generates a valuation in which each atomic formula is either true or false. (some subtlety here: more on this later)
\end{itemize}
\end{frame}

\begin{frame}[fragile]
\frametitle{Simple Predicates}
Here's what the definition of the \verb|shot| predicate (and related types and objects) might look like:
\begin{lstlisting}
type cutout
  is visible
  is sizable
  ...

object SmallCutoutN
  is cutout
  ...

pred shot[T]
  T cutout
\end{lstlisting}
\end{frame}

\begin{frame}[fragile]
\frametitle{Goals}
Idea: use formulas to define goals, and sets of goals (with point values) to define missions.
\lstset{moredelim=**[is][\color{red}]{@}{@}}
\lstset{escapeinside={(*}{*)}}
\begin{lstlisting}
(*\color{red}\textit{goal}*) @"Shot small cutouts in order."@
  (*\color{red}\textit{do}*)
    before(shot[(*\underline{SmallCutoutN}*)] and
           forall((*\underline{C}*) cutout, not(shot[(*\underline{C}*)]) or
                  (*\underline{C}*) = (*\underline{SmallCutoutN}*))) and
    shot[(*\underline{SmallCutoutW}*)]
  (*\color{red}\textit{for}*) @3000@
\end{lstlisting}
\end{frame}

\begin{frame}[fragile]
\frametitle{Actions and Rules}

\begin{tabular}[t]{l l}
\begin{lstlisting}
act actuate
  pre nil
  eff nil
  inv nil
  post
    actuated[A]
  eta 1s
  p 100%
...
\end{lstlisting}
&
\begin{lstlisting}
rule release
  X grabbable
  if
    grabbing[X]
  then
    actuated[UngrabActuator]
  implies
    ¬grabbing[S]
  p 100%
\end{lstlisting}
\end{tabular}
\end{frame}

\begin{frame}
\frametitle{Actions and Rules}
\item An \textit{action} is a set consisting of the following:
\begin{itemize}
\item \textit{preconditions}: formulas which must be satisfied in order for the action to be \textit{applicable},
\item \textit{effects}: formulas which \textit{may} become true \textit{during} the execution of the action (this is tricky! And unfortunately beyond the scope of this presentation),
\item \textit{invariants}: formulas which \textit{must} remain true \textit{during} the execution of the action,
\item \textit{postconditions}: formulas which the action \textit{asserts} will become true following its completion,
\item a \textit{time estimate}: an estimate of how long the action will take,
\item a \textit{success probability}: an estimate of how likely the action is to succeed (the last two form perhaps a distribution), and
\item an \textit{implementation}.
\end{itemize}
\end{frame}

\begin{frame}[fragile]
\frametitle{Actions and Rules}
An action describes a \textit{procedure} by which something can be accomplished.
\begin{lstlisting}
act actuate
  pre nil
  eff nil
  inv nil
  post
    actuated[A]
  eta 1s
  p 100%
\end{lstlisting}
\end{frame}

\begin{frame}[fragile]
\frametitle{Actions and Rules}
\textit{Rules} are preferred to actions, and describe procedures that can be themselves described in terms of \textit{sub-goals}.
\begin{lstlisting}
rule release
  X grabbable
  if
    grabbing[X]
  then
    actuated[UngrabActuator]
  implies
    not(grabbing[S])
  p 100%
\end{lstlisting}
\end{frame}

\begin{frame}[fragile]
\frametitle{Objects and Types}
\begin{lstlisting}
type stack
  is grabbable
  is visible
  is sizable
  is half_orientable

  # Gives us:
  # pred know[stack; color=C]
  #   C color
  color color
\end{lstlisting}
\begin{itemize}
\item An \textit{object} is a symbol associated with some collection of facts.
\item Essentially syntactic sugar for the \verb|know| predicate.
\item Important: means that all of the rest of what we're about to say about formulas applies to \verb|know| (object information) as well.
\end{itemize}
\end{frame}

\begin{frame}[fragile]
\frametitle{Quantifiers}
In the recovery mission, we only want to:
\begin{itemize}
\item only remove horizontal stacks if all vertical stacks have been removed, and
\item only decide on what horizontal stack to remove if there is some unremoved stack that we know is horizontal.
\end{itemize}
\lstset{moredelim=**[is][\color{red}]{@}{@}}
\begin{lstlisting}
act choose_horizontal_stack
  ...
  pre
    @exists@(S stack,
           not(removed[S]) and
           know[S; orientation=Horizontal]) and
    @forall@(S stack,
           know[S; orientation=Vertical]
           => removed[S])
  ...
\end{lstlisting}
\end{frame}

\begin{frame}[fragile]
\frametitle{Logical operators}
\lstset{moredelim=**[is][\color{red}]{@}{@}}
Logical operators allow us to create compound formulas out of atomic formulas.
\begin{lstlisting}
act choose_horizontal_stack
  ...
  pre
    exists(S stack,
           not(removed[S]) @and@
           know[S; orientation=Horizontal]) @and@
    forall(S stack,
           know[S; orientation=Vertical]
           @=>@ removed[S])
  ...
\end{lstlisting}

We have the standard operators from first-order logic: \verb|and|, \verb|or|, \verb|=>| (implication), and \verb|not|.
\end{frame}

\begin{frame}[fragile]
\frametitle{Temporal operators}
\begin{itemize}
\item An action can cause some formula's value to change...
\item ... but we might not want some things to change (or we might only care if some things have been true in the past).
\item A formula is evaluated in some temporal context (e.g. $t = 1487891636$).
\item We define three temporal operators:
\begin{itemize}
\item \verb|as(a, b)| is true when \verb|a| is true, and at some time that \verb|a| became true, \verb|b| was also true.
\item \verb|before(e)| is true when \verb|e| was true at some $t' \leq t$.
\item \verb|after(e)| is true when \verb|e| becomes true at some $t' \geq t$.
\end{itemize}
\item Time is lexically scoped (haha).
\end{itemize}
\end{frame}

\begin{frame}[fragile]
\frametitle{Temporal operators}
\lstset{moredelim=**[is][\color{red}]{@}{@}}
\lstset{escapeinside={(*}{*)}}
\begin{lstlisting}
rule drop_on_region
  if
    exists(S stack, R region, C color,
           ...)
  then
    not(grabbing[S])
  implies
    not(@after((thrusting{>= 10 N} and
               positioned{R - Sub =
                          (|xyz| < 1m)}) or
               positioned{R - Sub =
                          (|xyz| < 0.5m)})@)
    => on_table[S]
  p 100%
\end{lstlisting}
\end{frame}

\begin{frame}[fragile]
\frametitle{Algebraic predicates}
\begin{itemize}
\item Predicates so far have been applied to atomic objects; what about continuous values? 
\lstset{escapeinside={(*}{*)}}
\begin{lstlisting}
thrusting{>= 10 N} and
positioned{r - Sub = (|xyz| < 1m)})
\end{lstlisting}
\item How do we plan for compound formulas with continuous values? 
\item Complex predicates (within curly brackets) small languages to describe continuous constraints. 
\item Pull conjunctions and disjunctions into compound formulas....
\item ... then we can dispatch \verb|positioned| formulas, for example, to the motion planner as a single constraint.
\item {\tiny Algebraic because the predicate is acting as a homomorphism between Boolean algebras.}
\end{itemize}
\end{frame}

\begin{frame}[fragile]
\frametitle{Verifiers}
\begin{itemize}
\item How do we know whether an action actually succeeded?
\item \verb|verifiable(e)| is true when \verb|e| is currently believed to be \textit{verifiable}.
\end{itemize}
\begin{lstlisting}
rule position_above_tower_to_verify
  S stack
  if
    positioned{Tower - Sub =
               (|xy| < 10%fov, z > 2m, z < 3m)}
  then
    verifiable(removed[S])
  p 80%
\end{lstlisting}
\begin{itemize}
\item Verifiers implementations run in the background, returning whether relevant atomic formulas with \verb|true|, \verb|false|, or \verb|unknown|.
\item When to run verifiers (how much to trust actions?) characterizes part of the behavior of the mission controller with regard to risk.
\end{itemize}
\end{frame}

\begin{frame}
\frametitle{Other Topics}
\begin{itemize}
\item Mission planning \& re-planning.
\item Motion planning.
\item Learning.
\item Partial ordering of estimates.
\end{itemize}
\end{frame}

\end{document}

